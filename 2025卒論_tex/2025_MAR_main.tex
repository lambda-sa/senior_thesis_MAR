\documentclass[fleqn,12pt]{ltjsarticle}
% 他のタイプセットエンジンを使ってビルドする場合は, \documentclass[dvipdfmx]{jsarticle} などとする.
% パッケージの読み込み
\usepackage{comment}
\usepackage{graphicx}
\usepackage{color}
\usepackage{float}
\usepackage{listings}
\usepackage{amssymb, amsmath}
\usepackage{longtable}
\usepackage{subcaption}
\usepackage{hyperref}

% Document margins
\usepackage[a4paper, left=25mm, right=10mm, top=25mm, bottom=15mm]{geometry}

\setlength{\mathindent}{2em}
% Line spacing
\usepackage[labelformat=simple]{caption}
\captionsetup[table]{labelsep=space}
\captionsetup[figure]{labelsep=space}
\counterwithin{figure}{section}
\counterwithin{table}{section}

% Header and Footer (コメントアウトの指示を維持)
%\pagestyle{fancy}
%\fancyhead{} % Clear all header fields
%\fancyfoot{} % Clear all footer fields
%\fancyhead[L]{航空宇宙機設計に関する報告書}
%\fancyhead[R]{\thepage}
%\fancyfoot[C]{}
%\renewcommand{\headrulewidth}{0.4pt}
%\renewcommand{\footrulewidth}{0pt}
% === 修正終了 ===


\setcounter{table}{-1}

%\setcounter{section}{-1}
\setcounter{secnumdepth}{3}%見出しの番号をどの深さまで入れるか

\renewcommand{\figurename}{Fig.}
\renewcommand{\tablename}{Table}
\makeatletter
\renewcommand{\section}{\@startsection{section}{1}{\z@}%
    {1.5\Cvs \@plus.5\Cdp \@minus.2\Cdp}%
    {.5\Cvs \@plus.3\Cdp}%
    {\reset@font\centering\Large\bfseries}}
\makeatother


%表紙用の大きなフォントの定義
\def\HUGE{\fontsize{18pt}{22pt}\selectfont}

\begin{document}

\begin{titlepage}
\begin{center}
\vspace*{8\baselineskip}
2025年度
\vspace*{2\baselineskip}

\HUGE
航空機設計仕様書
\vspace*{9\baselineskip}

2025年8月
\vspace*{2\baselineskip}

東京理科大学創域理工学部機械航空宇宙工学科
\vspace*{1\baselineskip}

小笠原研究室
\vspace*{3\baselineskip}

\begin{tabular}{ll}
7522095 & 舟木 悠太 \\

\end{tabular}
\end{center}
\end{titlepage}

% 目次
\setcounter{tocdepth}{3}
\tableofcontents

\clearpage

\section{序論}
\subsection{研究背景}
%近年,宇宙開発の分野において,安全かつ効率的な宇宙飛行士の地球帰還手段の確立が求められている.特に,緊急時における迅速な帰還能力は,宇宙ミッションの成功と宇宙飛行士の安全確保に不可欠である.このような背景から,パラフォイルを用いた宇宙飛行士の地球帰還システムが注目されている.パラフォイルは,従来のパラシュートに比べて高い制御性と滑空性能を有しており,精密な着陸地点への誘導が可能である.そのため,パラフォイルを活用した帰還システムは,宇宙飛行士の安全性向上とミッション成功率の向上に寄与すると期待されている.本研究では,パラフォイルを用いた宇宙飛行士の地球帰還システムの設計と最適化を目的とし,その性能評価と実用化に向けた課題解決を目指す.
\subsection{パラフォイルの仕組み}
この節では,パラフォイルの基本的な仕組みについて説明する.パラフォイルは,展開可能な柔軟構造を有し,端部の形状を変形させることで姿勢の変更及び軌道制御が可能な落下傘を指す(図(\ref{fig:para_real})).
\begin{figure}[H]
  \centering
  \includegraphics[width=0.6\linewidth]{画像/パラフォイル_実機正面図.jpeg}
  \caption{The X-38 prototype of the Crew Return Vehicle is suspended under its giant 7,500-square-foot parafoil during its eighth free flight on Thursday, December 13, 2001.\cite{ref_NASA}}
  \label{fig:para_real}
\end{figure}
パラフォイルはキャノピー,ペイロード,そしてテザーから構成されている.パラフォイルのキャノピーは,翼型断面を有し,揚力を発生させることで滑空飛行を実現する.ペイロードは,パラフォイルに吊り下げられた物体であり,テザーはキャノピーとペイロードを接続する.
パラフォイルは前進時にキャノピーが空気を取り込み,空気流のせきとめ圧(ラム圧)でセルを膨らませて翼形状を保持する.パラフォイルは減速や降下に用いられるパラシュートと比べてアスペクト比が大きく,端面が翼型形状であるため滑空飛行が可能となる.\\
パラフォイルの制御は,主にテザーの端部を引くことで行われ、テザーの片端を引く非対称制御とテザーの両端を引く対称制御の2通りの制御方法を使い分ける.非対称制御では,例としてテザーの右側を引くとキャノピーの右後部が下がることでバンク角が発生し、向心力が働くことで左旋回が可能となる.対称制御では,両端のテザーを引くことでキャノピーの後縁が下がり揚抗比が変化することで前進速度を制御できる.

%テザーの端部を引っ張ると,キャノピーの形状が変化し,揚力の分布が変わることで姿勢や進行方向を制御できる.例えば,右側のテザーを引っ張ると,右側の揚力が増加し,左側に旋回することが可能となる.


\subsection{先行研究}
\subsubsection{3GMAR(3rd Generation Mid-Air Retrieval )の手順}
空中回収において,ヘリコプターとパラフォイルが安定した会合を行うため,会合時の衝撃荷重及び相対速度の低減及び空中回収に適したパラフォイルの種類の研究が行われてきた.
Deanらは,3GMAR(3rd Generation Mid-Air Retrieval )という空中回収の方式を提案した.3GMARの手順は以下の通りである(図(\ref{fig:para_real})).
\begin{figure}[H]
  \centering
  \includegraphics[width=0.8\linewidth]{画像/3GMARの概念図.jpg}
  \caption{The process of 3GMAR.\cite{ref_mar_past_pre}}
  \label{fig:3gMAR}
\end{figure}
3GMARではフックが取り付けられた回収ヘリコプターが,ドローグパラシュートが取り付けられたパラフォイルに接近し,ドローグにフックをかけて回収する.
この時の手順は接近,会合,引き上げの3段階に分けられる.接近段階では,ヘリコプターは、パラフォイルをパイロットが目視で確認できるまでパラフォイルに向かって飛行する.
その後,回収ヘリコプターはパラフォイルと右斜め編隊を組み,パラフォイルより約15m高い高度を維持し,パラフォイルの中心線の左側に約15m離れて位置する.このときヘリコプターとパラフォイルとの相対速度はほぼ0m/sとなる.
会合段階では,ヘリコプターはパラフォイルに向かって徐々に接近し,ドローグパラシュートにフックをかける.
引き上げ段階では,ヘリコプターはドローグパラシュートを引き上げる.この際,係合ラインに張力がかかり,パラフォイルのサスペンションラインに取り付けられた「スライダー」と呼ばれる部品が引き上げられキャノピーが収縮する.これにより,パラフォイルの揚力が減少し,ヘリコプターへの負荷が軽減される.\\
3GMARでは,パラフォイルがヘリコプターの前方を滑空するため,ヘリコプターのパイロットがパラフォイルを追跡するのが容易である上,ヘリコプターとパラフォイルの相対速度が小さいため,会合時の衝撃荷重が低減される.これにより,安全かつ効率的な空中回収が可能となる.
その一方,3GMARではパラフォイルが自律制御を行うことは想定されていないため,風などの外乱やパラフォイルの初期値のずれによりパラフォイルがヘリコプターから離れた方向に移動する可能性がある.したがって空中回収の成功は、ヘリコプターの性能及びパイロットの技術に依存している.(\cite{ref_mar_past_pre})


\begin{thebibliography}{1}
%\bibitem{ref1} 李家 賢一,航空機設計法 実践編
%- 小型ジェット旅客機からハイブリッド電動航空機の概念設計まで -,pp. 14―32.
\bibitem[Dean,2005]{ref_mar_past_pre}The Past, Present, and Future of Mid-Air Retrieval
Dean S. Jorgensen•, Roy A. Haggardt and Glen J. Brown Vertigo, Inc., Lake Elsinore, California 92531, 18th AIM Aerodynamic DeceleratorS ystems Technology Conference and Seminar(2005)
\bibitem[NASA,2001]{ref_NASA}The National Aeronautics and Space Administration, The X-38 prototype of the Crew Return Vehicle is suspended under
its giant 7,500-square-foot parafoil during its eighth free flight on Thursday, December 13, 2001, from
<https://www.nasa.gov/image-detail/amf-ec01-0339-146/>, (参照日2024年1月2日)
\end{thebibliography}
\end{document}