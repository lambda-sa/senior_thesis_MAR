\documentclass[12pt]{ltjsarticle}%長文ならltjsreport
%\documentclass[a4j]{jarticle} %ここは関係ない
% ltjsarticle: lualatex 用の 日本語 documentclass
% 他のタイプセットエンジンを使ってビルドする場合は, \documentclass[dvipdfmx]{jsarticle} などとする.
% パッケージの読み込み
% \usepackage[dvipdfmx]{graphicx}
%\usepackage{listings,jvlisting} 
\usepackage{graphicx}
\usepackage{color}
\usepackage{float}
\usepackage{listings}
\usepackage{amssymb, amsmath}
\usepackage[top=20truemm,bottom=15truemm,left=25truemm,right=10truemm]{geometry}
\usepackage{longtable}

\usepackage{graphicx}
\usepackage{subcaption}
%\usepackage[inline]{enumitem}
%\usepackage{listings,jvlisting} 
%\usepackage{lipsum} % ダミーテキストを挿入するため


\lstset{
	%プログラム言語(複数の言語に対応,C,C++も可)
 	language = Python,
 	%背景色と透過度
 	backgroundcolor={\color[gray]{.90}},
 	%枠外に行った時の自動改行
 	breaklines = true,
 	%自動開業後のインデント量(デフォルトでは20[pt])	
 	breakindent = 10pt,
 	%標準の書体
 	basicstyle = \ttfamily\scriptsize,
 	%basicstyle = {\small}
 	%コメントの書体
 	commentstyle = {\itshape \color[cmyk]{1,0.4,1,0}},
 	%関数名等の色の設定
 	classoffset = 0,
 	%キーワード(int, ifなど)の書体
 	keywordstyle = {\bfseries \color[cmyk]{0,1,0,0}},
 	%""で囲まれたなどの"文字"の書体
 	stringstyle = {\ttfamily \color[rgb]{0,0,1}},
 	%枠 "t"は上に線を記載, "T"は上に二重線を記載
	%他オプション:leftline,topline,bottomline,lines,single,shadowbox
 	frame = TBrl,
 	%frameまでの間隔(行番号とプログラムの間)
 	framesep = 5pt,
 	%行番号の位置
 	numbers = left,
	%行番号の間隔
 	stepnumber = 1,
	%右マージン
 	%xrightmargin=0zw,
 	%左マージン
	%xleftmargin=3zw,
	%行番号の書体
 	numberstyle = \tiny,
	%タブの大きさ
 	tabsize = 4,
 	%キャプションの場所("tb"ならば上下両方に記載)
 	captionpos = t
}

\setcounter{section}{-1}
\setcounter{secnumdepth}{1}
\renewcommand{\figurename}{Fig.}
\renewcommand{\tablename}{Table}


\makeatletter
\newcommand{\affiliation}{\gdef\@affiliation}
\affiliation{}
\makeatother

\makeatletter


\renewcommand{\maketitle}{
  \begin{center}
    {\large\bfseries\@title}
    \vspace{0.5cm}
  \end{center}

  \noindent
  \begin{minipage}[t]{0.5\linewidth}
    \raggedright \@affiliation
  \end{minipage}%
  \begin{minipage}[t]{0.5\linewidth}
    \raggedleft \@author
  \end{minipage}

  \vspace{0.5cm}
  %{\@date}
}

	
	 
\renewcommand{\section}{\@startsection{section}{1}{\z@}%
   {1.5\Cvs \@plus.5\Cdp \@minus.2\Cdp}%
   {.5\Cvs \@plus.3\Cdp}%
   {\reset@font\centering\Large\bfseries}}       %センタリング..
	

\makeatother

\title{Vector Field 法によるパラフォイル空中回収に向けた誘導の研究}
\affiliation{[小笠原研究室]}
\author{7522095 舟木悠太}  

\begin{document}

\maketitle
宇宙開発の分野において, パラフォイルを展開して降下するペイロードをヘリコプターで捕獲する空中回収という方式の研究が行われている. %ペイロードの回収では, パラフォイルを展開して地上や水上に落下させる方式が一般的であるが, これらの方式は着陸または着水時にペイロードに大きな衝撃が加わる上, 落下したペイロードを回収するのに時間がかかる. 
空中回収はヘリコプターを用いてペイロードを空中で捕獲する方式であり, ペイロードに衝撃を与えずに迅速に回収できる利点があるため, %Genesisのサンプルリターンミッションなどで採用されており, 太陽風などの
衝撃に弱い物体を地球に持ち帰る手段として有効である. 

空中回収でのパラフォイルの誘導に関しては, VF法(Vector Field法)に基づく経路追従手法が提案されているが,  
空中回収に適した軌道計画がなされていない点及び, 制御パラメータ調整の煩雑な点が課題として挙げられる. %VF法では, 参照経路に対するパラフォイルの位置ずれを低減するために, 複数の制御パラメータを調整する必要がある. これらのパラメータは, 風速や風向きなどの外乱条件に応じて決定されるべきであるが, 
そこで, 本研究では,
空中回収に適した軌道計画手法の構築及び,VF法の修正によるパラフォイルの誘導則の評価を目的として6自由度の運動モデルによるシミュレーションを行った. 

シミュレーションにより, Dubins Pathで作成した軌道にクロゾイド曲線を緩和曲線として導入することでパラフォイルの誘導性能は向上する一方, ubins Path 単体の場合と比べて計算誤差が大き
いので, 参照軌道と会合点自体に誤差は生じた.また, VF法を修正することで,  ゲインチューニングの手間を大幅に減らしたう
えで, 空中回収を目的に計画された軌道に会合点との誤差が 4.50m になるまで追従させられ
た. その一方, VF 法には積分項が存在しないため, 横風が吹いたときに定常偏差が残り続ける結果となった. 

このことから, 今後の展望として軌道計画時にはクロゾイド曲線以外の緩和曲線を検討することと, VF 法で定常偏差を抑制する方法を考えることが挙げられる. 
\begin{figure}[H]
  \centering
  \includegraphics[width=0.8\linewidth]{画像/newVF_all_image.png}
  \caption{Concept of Vector Field method.}
  \label{fig:newVF_concept}
\end{figure}

% --- ここから英語セクション ---
\newpage

% 英語タイトルの手動作成 (プリアンブルの maketitle 定義と同様のレイアウト)
\begin{center}
  {Study on Guidance for Parafoil Mid-Air Retrieval using Vector Field Method}
  \vspace{0.5cm}
\end{center}

\noindent
\begin{minipage}[t]{0.5\linewidth}
  \raggedright [Ogasawara Group]
\end{minipage}%
\begin{minipage}[t]{0.5\linewidth}
  \raggedleft 7522095 Yuta FUNAKI
\end{minipage}

\vspace{0.5cm}

This study investigates guidance strategies for the Mid-Air Retrieval of parafoils using a helicopter. 
We developed a trajectory planning method integrating Dubins paths with Clothoid curves 
and evaluated a modified Vector Field (VF) method via 6-DOF simulations. 
While Clothoid curves improved guidance performance, they introduced calculation errors at the rendezvous point. 
The modified VF method significantly reduced tuning efforts, achieving a rendezvous error of 4.50 m. 
However, steady-state errors persisted under crosswinds due to the lack of an integral term. 
Future work will explore alternative transition curves and methods to mitigate steady-state deviations.

\end{document}
